
\newcommand\areaplot[3]{

% \pgfplotsset{every axis legend/.append style={
%   at={(-0.15,0)},
%   anchor=south east,
%   cells={anchor=west},
%   reverse legend}}

\pgfplotsset{every axis legend/.append style={
  at={(0.5,-0.20)},
  anchor=north,
  legend columns=-1,
 %  reverse legend
  }}

\begin{tikzpicture}
    \begin{semilogxaxis}[
        stack plots=y,
        area style,
        ymin=0,
        ymax=100,
        ylabel=tool distribution,
        yticklabel={$\pgfmathprintnumber{\tick}$\%},
        xlabel=time (s),
        xmin = 1,
        enlarge x limits=false,
        ]
    \addplot
        table[x=t, y expr=100*\thisrow{me}/\thisrow{sum}] {#1}
        \closedcycle;
    \addplot
        table[x=t, y expr=100*\thisrow{us}/\thisrow{sum}] {#1}
        \closedcycle;
    \addplot
        table[x=t, y expr=100*\thisrow{them}/\thisrow{sum}] {#1}
        \closedcycle;
    \legend{#2,both,#3}
    \end{semilogxaxis}
    \begin{semilogxaxis}[
        axis y line=right,
        ylabel=solved problems,
        ymin=0,
        ymax=400,
        axis x line=none,
        enlarge x limits=false]
    \addplot [black, thick]
        table[x=t, y=sum, mark=none] {#1};
    \end{semilogxaxis}
\end{tikzpicture}

}

% For this assignment then, I ask you to do two things. First, supply the
% ‘storyboard’ of data illustrations of your research for a specific article and
% then the main idea of the analysis. Then, I ask you to use any one illustration
% of data from the research as if you were using it in a research article or a
% conference paper.
%
% So, you supply the illustration, its caption, and the data commentary required
% for it to be effective in an article. If it is a figure or table you have used
% before or if it is new data matters less as long as you provide the data
% commentary necessary for the data to communicate effectively.
%
% The assignment is supposed to be useful to you and you must use it as you see
% fit, but it is intended as a unit that can serve as an example for you in
% future writing if you cannot use it immediately in your current writing.
%
% \begin{centering}
% symbo vs crappy
% \areaplot{symbo_crappy.dat}{symbo}{crappy}
% symbo vs feat
% \areaplot{symbo_feat.dat}{symbo}{feat}
%
% \newpage
% symbo vs smten
% \areaplot{symbo_smten.dat}{symbo}{smten}
% crappy vs smten
% \areaplot{crappy_smten.dat}{crappy}{smten}
%
% \newpage
% symbo vs lazy smallcheck
% \areaplot{symbo_lazysc.dat}{symbo}{lazy sc}
% crappy vs lazy smallcheck
% \areaplot{crappy_lazysc.dat}{crappy}{lazy sc}
% \end{centering}

\subsection*{Data commentary}
\vspace{-0.5\baselineskip}
\emph{Dan Rosén}
\vspace{0.5\baselineskip}

Figures~\ref{vs-smten} and ~\ref{vs-lazysc}
compares our tool with smten and LazySmallCheck, respectivelly.
%
In the experiments, the completion runtime was recorded
for each tool and benchmark in the test suite.
At any given time there are some benchmarks solved
by both tools and some only solved by one of them.
The graphs are divided into these three segments,
each of them indicating this distribution proportionally.
There is also an overlayed line telling how many benchmarks were solved.
Time on the x-axis is logarithmic and goes from 1 to 60 seconds.

In the comparison with smten in Figure~\ref{vs-smten}
we see that HBMC solves a bigger chunk of benchmarks than smten.
The numbers of benchmarks these solvers solve together are less than 300.
When comparing to LazySmallCheck in Figure~\ref{vs-lazysc},
they together solve more than 300 benchmarks.
HBMC's sole contribution is initially not so significant,
but its share over time increases significantly.
This indicates that HBMC is an important contribution
to solving the most difficult and challenging benchmarks.

\begin{figure}[H]
\begin{centering}
\areaplot{crappy_smten.dat}{HBMC}{smten}
\end{centering}
\caption{HBMC compared with smten. \label{vs-smten}}
\end{figure}

\begin{figure}[H]
\begin{centering}
\areaplot{crappy_lazysc.dat}{HBMC}{LazySmallCheck}
\end{centering}
\caption{HBMC compared with LazySmallCheck. \label{vs-lazysc}}
\end{figure}


